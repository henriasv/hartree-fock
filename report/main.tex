\documentclass[a4paper,10pt]{article}
\usepackage[utf8]{inputenc}

\usepackage{amsmath}
\usepackage{caption}

\newcommand{\rvec}{\mathbf{r}}
\newcommand{\Rvec}{\mathbf{R}}
\newcommand{\xvec}{\mathbf{x}}
\newcommand{\dd}{\mathrm{d}}
\newcommand{\Pvec}{\mathbf{P}}

%opening
\title{Project - FYS4411}
\author{Henrik Andersen Sveinsson}

\begin{document}

\maketitle

\begin{abstract}

\end{abstract}

\section{The Hartree-Fock method}

\subsection{Approximations}
There are five main approximations in the Hartree-Fock method:
\begin{itemize}
 \item The Born-Oppenheimer approximation
 \item Relativistic effects are neglected
 \item The solution is a linear combination of (typically orthogonal) basis functions
 \item Each energy eigenfunction is assumed to be describable by a single Slater determinant
 \item The mean field approximation is implied. 
\end{itemize}

The Hartree equation in atomic units:
\begin{equation}
 \left[ -\frac{1}{2}\nabla^2 -\sum_n \frac{Z_n}{|\rvec -\Rvec_n|} + 
 \sum_{l=1}^N \int \dd x' |\psi_l(\xvec)|^2 \frac{1}{|\rvec -\rvec'|}\right] \psi_k(\xvec) 
 = E' \psi_k (\xvec)
\end{equation}

Note that $\xvec' = (\rvec', s')$, so that $\int \dd x' = \sum_{s'} \int \dd \rvec'$.

The last term on the left side is called the \emph{Hartree potential}. 

\subsection{The Slater determinant}
Since electrons are indistinguishable, the Hamiltonian commites with the particle-exchange operator $P_{ij}$:
\begin{equation}
	\Pvec_{ij}\Psi(\xvec_1, \xvec_2, ..., \xvec_i, ..., \xvec_j, ... \xvec_N) = \Psi(\xvec_1, \xvec_2, ..., \xvec_j, ..., \xvec_i, ... \xvec_N)
\end{equation}

The eigenvalue of $P$ is $-1$ (experimental) for fermions. We are working with an independent-particle Hamiltonian, so the electron state can be written as a product of single-electron states:
\begin{equation}
	\Psi(\xvec_1, ..., \xvec_N) = \psi_1(\xvec_1)\cdots\psi_N(\xvec_N)
\end{equation}

\end{document}
